%!TeX program=lualatex
%!TeX spellcheck=en_US
\documentclass[10pt,lualatex,xcolor={table,svgnames},{hyperref={bookmarks=true,linktoc=all}},aspectratio=169]{beamer}
\usetheme{o2report}
\usepackage{amssymb,amsmath}
\usepackage{graphicx}
\graphicspath{{figures/}}
\usepackage{xcolor}
\usepackage{colortbl}
\usepackage{xltxtra}
\usepackage{fontspec}
\usepackage{unicode-math}
\usepackage{appendixnumberbeamer}
\usepackage{wasysym}

\usepackage{csquotes}
\usepackage{pbox}
\usepackage{xhfill}
\usefonttheme{professionalfonts}
\setmainfont{Source Serif Pro}%{Cambria}%
\setsansfont{Source Sans Pro}%{Cambria}%
\defaultfontfeatures{}
\setmonofont{Source Code Pro}%{Cambria}%
\setmathfont{Cambria Math}
%\usepackage{mathastext}
\usepackage{bm,tabu,booktabs,multirow}
\usepackage{ragged2e}
\usepackage{multicol}

\tolerance=100
\setcounter{tocdepth}{3}

\usepackage{xfrac}
\usepackage{xspace,indentfirst}%,paralist}

\setlength{\intextsep}{2pt plus 3pt}% Vertical space above & below [h] floats
\setlength{\textfloatsep}{2pt plus 3pt}% Vertical space below (above) [t] ([b]) floats
\setlength{\abovecaptionskip}{2pt plus 3pt}%
\setlength{\belowcaptionskip}{2pt plus 1pt}%

\usepackage{etoolbox}
\appto\normalsize{%
	\abovedisplayskip=0pt
	\belowdisplayskip=0pt
	\abovedisplayshortskip=0pt
	\belowdisplayshortskip=0pt
}
\appto\small{%
	\abovedisplayskip=0pt
	\belowdisplayskip=0pt
	\abovedisplayshortskip=0pt
	\belowdisplayshortskip=0pt
}

\usepackage[permil]{overpic}

\usetikzlibrary{arrows.meta,graphs,graphdrawing}
\usegdlibrary{layered,trees,force}

%\usetikzlibrary{fit,calc,backgrounds,arrows}
\usepackage{endnotes}
\usepackage{minted}[breaklines=true,bgcolor=LightGray]
\usepackage{lipsum}

\newcommand{\logical}[1]{\textcolor{Green}{#1}}
\newcommand{\programmatic}[1]{\textcolor{-green!40!yellow}{#1}}
\newcommand{\notion}[1]{\textcolor{teal}{#1}}
\newcommand{\operation}[1]{\textcolor{FireBrick}{#1}}

\newcommand{\codelines}[2]{{\mintinline[bgcolor=gray!20,fontsize=#2]{cpp}|#1|}}
\newcommand{\codeline}[1]{{\mintinline[bgcolor=gray!20,fontsize=\small]{cpp}|#1|}}

\usemintedstyle{rainbow_dash}

\begin{document}
    \author[Anton Alkin]{Anton Alkin}
    \title{O\textsuperscript{2} hands-on session}
    \date[O\textsuperscript{2} Hackaton, 07/12/2021]{HF O\textsuperscript{2} Hackatons, 07/12/2021}

    \begin{frame}{}
        \maketitle
    \end{frame}

    \section{Introduction}

    \begin{frame}{Outline}
        \begin{columns}
            \begin{column}{0.02\paperwidth}
            \end{column}
            \begin{column}[c]{0.9\paperwidth}
                \begin{itemize}
                    \item {\huge\bfseries Adding a task}
                    \item {\huge\bfseries Getting the data, creating outputs}
                    \item {\huge\bfseries Filters, Partitions, Index tables, workflows}
                \end{itemize}
                \vspace{1em}
                {\small
                    \begin{description}
                        \item[Official documentation] \href{https://aliceo2group.github.io/analysis-framework/docs/framework/}{https://aliceo2group.github.io/analysis-framework/docs/framework/}
                \end{description}}
            \end{column}
        \end{columns}
    \end{frame}

   \begin{frame}[fragile]{O2Physics build system}
   \begin{block}{CMakeLists.txt}
       \begin{minted}{cmake}
o2physics_add_dpl_workflow(histograms
                           SOURCES src/histograms.cxx
                           PUBLIC_LINK_LIBRARIES O2::Framework
                           COMPONENT_NAME AnalysisTutorial)
       \end{minted}
   \end{block}
    \begin{block}{Adding a task}
        \begin{enumerate}
            \item Add a source file in an appropriate folder (we'll be using \verb|Tutorials|)
            \item Add an entry in the corresponding \verb|CMakeLists.txt|
            \item Recompile
        \end{enumerate}
    \end{block}
   \end{frame}

\begin{frame}[fragile,allowframebreaks]{Adding a task}
    \begin{description}
        \item[Add a source]
            \texttt{Tutorials/src/tutorial1.cxx}
            \begin{minted}[fontsize=\scriptsize]{cpp}
// Copyright 2019-2020 CERN and copyright holders of ALICE O2.
// See https://alice-o2.web.cern.ch/copyright for details of the copyright holders.
// All rights not expressly granted are reserved.
//
// This software is distributed under the terms of the GNU General Public
// License v3 (GPL Version 3), copied verbatim in the file "COPYING".
//
// In applying this license CERN does not waive the privileges and immunities
// granted to it by virtue of its status as an Intergovernmental Organization
// or submit itself to any jurisdiction.

#include "Framework/runDataProcessing.h"
#include "Framework/AnalysisTask.h"

using namespace o2;
using namespace o2::framework;

struct TutorialOne {
};

WorkflowSpec defineDataProcessing(ConfigContext const& cfgc)
{
    return WorkflowSpec{adaptAnalysisTask<TutorialOne>(cfgc)};
}
            \end{minted}
        \item[Add an entry]
        \begin{minted}[fontsize=\scriptsize]{cmake}
o2physics_add_dpl_workflow(tutorial1
                           SOURCES src/tutorial1.cxx
                           PUBLIC_LINK_LIBRARIES O2::Framework
                           COMPONENT_NAME AnalysisTutorial)
        \end{minted}
    \item[Recompile]
    \begin{minted}[fontsize=\footnotesize]{shell}
% cd $ALIBUILD_WORK_DIR/BUILD/O2Physics-latest/O2Physics
(direnv stuff)
% ninja -j16 install
(compilation)
    \end{minted}
\item[Or partially]
\begin{minted}[fontsize=\footnotesize]{shell}
% ninja -j16 stage/bin/o2-analysistutorial-tutorial1
(compilation)
\end{minted}
    \end{description}
\end{frame}

\begin{frame}[fragile]{Running a task}
\begin{block}{Regular way}
\begin{minted}[fontsize=\footnotesize]{shell}
% alienv enter O2Physics/latest
(init stuff)
% o2-analysistutorial-tutorial1 --aod-file path/to/AO2D.root
\end{minted}
\vspace{0.5ex}
\end{block}
\vspace{1em}
\begin{block}{Quick testing}
\begin{minted}[fontsize=\footnotesize]{shell}
% ninja -j16 stage/bin/o2-analysistutorial-tutorial1
(compilation)
% stage/bin/o2-analysistutorial-tutorial1 --aod-file path/to/AO2D.root
\end{minted}
\vspace{0.5ex}
\end{block}
\end{frame}

\begin{frame}[fragile]{Misc.}
\begin{block}{\enquote{Solutions}}
\begin{minted}[fontsize=\footnotesize]{shell}
% git remote add aalkin https://github.com/aalkin/O2Physics
% git fetch aalkin
% git checkout aalkin/hands-on-examples
% ls Tutorials/src/example_*.cxx
Tutorials/src/example_1.cxx Tutorials/src/example_3.cxx Tutorials/src/example_5.cxx
Tutorials/src/example_2.cxx Tutorials/src/example_4.cxx Tutorials/src/example_6.cxx
Tutorials/src/example_7.cxx
\end{minted}
\vspace{0.5ex}
\end{block}
\begin{block}{Input file}
LHC18g4, pp@13~TeV, simulation \\
\href{http://alitrain.cern.ch/train-workdir/PWGZZ/Run3_Conversion/232_20211108-1726/test/__ALL__/AO2D.root}{http://alitrain.cern.ch/train-workdir/PWGZZ/Run3\_Conversion/232\_20211108-1726/test/\_\_ALL\_\_/AO2D.root}
\end{block}
\end{frame}

\section{Examples}

\begin{frame}[fragile]{Example 1}
\begin{block}{Access tracks}
\textbf{Print out track transverse momenta}
\end{block}
\begin{block}{Process function}
    \begin{minted}[fontsize=\footnotesize]{cpp}
void process(aod::Tracks const& tracks) {}
    \end{minted}
\end{block}
\begin{block}{Loop over tracks}
\begin{minted}[fontsize=\footnotesize]{cpp}
for (auto& track : tracks) {
    LOGP(info, "Track {} has pT = {}", track.index(), track.pt());
}
\end{minted}
\end{block}
{\footnotesize Relevant tutorial:  \href{https://github.com/AliceO2Group/O2Physics/blob/master/Tutorials/src/trackIteration.cxx}{https://github.com/AliceO2Group/O2Physics/blob/master/Tutorials/src/trackIteration.cxx} \\
Complete example: \href{https://github.com/aalkin/O2Physics/blob/hands-on-examples/Tutorials/src/example_1.cxx}{https://github.com/aalkin/O2Physics/blob/hands-on-examples/Tutorials/src/example\_1.cxx}
}
\end{frame}

\begin{frame}[fragile,allowframebreaks]{Example 2}
\begin{block}{Add a histogram}
\textbf{Add a $p_{\text{T}}$ histogram with range from 0 to 10, equal bins with width of 0.1}
\end{block}
\begin{block}{Histogram registry}
\begin{minted}[fontsize=\footnotesize]{cpp}
HistogramRegistry registry{
    "registry",
    {
        {"name", "title ; axis", {HistType::kTH1F, {{n, min, max}}}}
    }
};
\end{minted}
\end{block}
\begin{block}{Accessing histograms in registry}
\begin{minted}[fontsize=\footnotesize]{cpp}
// get a pointer
auto* histo_ptr = registry.get<TH1>(HIST("name"));
// fill single value directly - a direct wrapper for TH1::Fill
registry.fill(HIST("name"), value, weight);
// fill a collection using a filter
registry.fill<aod::track:Pt>(HIST("name"), input_table, condition);
\end{minted}
\end{block}
\framebreak
\begin{block}{Simple filling}
\begin{minted}[fontsize=\footnotesize]{cpp}
for (auto& track : tracks) {
    registry.fill(HIST("hpt"), track.pt());
}
\end{minted}
\end{block}
\begin{block}{Check the output}
\begin{minted}[fontsize=\footnotesize]{shell}
% root -l AnalysisResults.root
root [0]
Attaching file AnalysisResults.root as _file0...
(TFile *) 0x7ffb94bbb5e0
root [1] .ls
TFile**		AnalysisResults.root
TFile*		AnalysisResults.root
KEY: TDirectoryFile	example-two;1	example-two
root [2]
\end{minted}

\textcolor{red}{AnalysisResults.root \emph{will be overwritten} each time you run a workflow}
\end{block}

{\footnotesize Relevant tutorial:  \href{https://github.com/AliceO2Group/O2Physics/blob/master/Tutorials/src/histogramRegistry.cxx}{https://github.com/AliceO2Group/O2Physics/blob/master/Tutorials/src/histogramRegistry.cxx} \\
    Complete example: \href{https://github.com/aalkin/O2Physics/blob/hands-on-examples/Tutorials/src/example_2.cxx}{https://github.com/aalkin/O2Physics/blob/hands-on-examples/Tutorials/src/example\_2.cxx}
}
\end{frame}

\begin{frame}[fragile,allowframebreaks]{Example 3}
\begin{block}{Add another histogram}
\textbf{Add an event average $p_{\text{T}}$ histogram with range from 0 to 2, equal bins with widths of 0.1}
\end{block}
\begin{block}{Advanced data subscription: grouping}
Now we need to access tracks per collision
\begin{minted}[fontsize=\footnotesize]{cpp}
void process(aod::Collision const& collision, aod::Tracks const& tracks) {}
\end{minted}
\notion{Note 1}: \codeline{Collision} without \enquote{s} \\
\notion{Note 2}: in \codeline{AnalysisDataModel.h} we have \codeline{using Collision = Collisions::iterator;} \\
\notion{General rule}: if the \emph{first} argument of the process function is an \programmatic{iterator}, all subsequent arguments, that have an \programmatic{index} connection to the first, will be \programmatic{grouped}
\end{block}

{\footnotesize Relevant tutorial:  \href{https://github.com/AliceO2Group/O2Physics/blob/master/Tutorials/src/collisionTracksIteration.cxx}{https://github.com/AliceO2Group/O2Physics/blob/master/Tutorials/src/collisionTracksIteration.cxx} \\
    Complete example: \href{https://github.com/aalkin/O2Physics/blob/hands-on-examples/Tutorials/src/example_3.cxx}{https://github.com/aalkin/O2Physics/blob/hands-on-examples/Tutorials/src/example\_3.cxx}
}
\end{frame}

\begin{frame}[fragile,allowframebreaks]{Example 4}
\begin{block}{Acessing generator-level information}
\textbf{Add a $p_{\text{T}}$ and event average $p_{\text{T}}$ histograms at generator level} \\[1em]
\notion{Note 1}: We need either a separate task or an additional proccess function with different subscriptions \\
\notion{Note 2}: Additional process functions require control switches \\
\notion{Note 3}: Only one function with a name \codeline{process()} is allowed and it is executed unconditionally \\
\notion{General rule}: Each process function (that is not named \codeline{process()}) needs a corresponding \codeline{PROCESS_SWITCH} declaration
\end{block}
\begin{block}{Configurables, Process switches}
\textbf{Add an $\eta$ cut for tracks} \\[1em]
\programmatic{Process switch} is a special case of a more generic \programmatic{Configurable} -- a variable that is set externally \\
\notion{Note 1}: \codeline{Configurable<type> varname{"configName", default_value, "Description"};} \\
\notion{Note 2}: \codeline{PROCESS_SWITCH(TaskType,processFunction,"Description",default_state);} \\
\end{block}
\framebreak
\begin{block}{Upfront filtering}
For cases were we only require a subsample of a collection (collisions, tracks, V0s, etc.) we can use expression-based filters \\

\codelines{Filter etaFilter = nabs(aod::track::eta) <= etaCut;}{\footnotesize}

This requires also declaring, which arguments should have filters applied to them \\
\codelines{void process(aod::Collision const& collision, soa::Filtered<aod::Tracks> const& tracks)}{\footnotesize}
\vspace{0.2ex}
\end{block}
\begin{block}{Additional process function}
    \begin{minted}[fontsize=\footnotesize]{cpp}
        void processMC(aod::McCollision const& mccollision,
        soa::Filtered<aod::McParticles> const& particles) {}
        // after the function
        PROCESS_SWITCH(ExampleFour, processMC, "Process MC info", false);
    \end{minted}
\end{block}
\framebreak
\begin{block}{Setting external configuration}
\begin{tabular}{p{0.3\linewidth}p{0.6\linewidth}}
{\begin{minted}[fontsize=\small,breaklines]{json}
{
    "example-four": {
        "etaCut": "0.5",
        "processMC": "true"
    }
}
\end{minted}
} &
{\begin{itemize}
    \item On exit workflow produces \codeline{dpl-config.json}
    \item It can be copied, edited, and used for subsequent runs
    \item \textcolor{red}{Process switches can only be changed via JSON}
\end{itemize}
} \\
\end{tabular}

\begin{minted}[fontsize=\small,bgcolor=gray!20]{shell}
% o2-analysistutorial-example-4 --configuration=json://example_4.json
\end{minted}
\end{block}
{\footnotesize Relevant tutorial:  \href{https://github.com/AliceO2Group/O2Physics/blob/master/Tutorials/src/multiProcess.cxx}{https://github.com/AliceO2Group/O2Physics/blob/master/Tutorials/src/multiProcess.cxx} \\
    Complete example: \href{https://github.com/aalkin/O2Physics/blob/hands-on-examples/Tutorials/src/example_4.cxx}{https://github.com/aalkin/O2Physics/blob/hands-on-examples/Tutorials/src/example\_4.cxx}
}
\end{frame}

\begin{frame}[fragile,allowframebreaks]{Example 5}
\begin{block}{Joining tables, building workflows}
\textbf{Replace event average $p_{\text{T}}$ histograms with 2-dimensional vs. V0M multiplicity}\\[1em]

\notion{Note 1}: We now require extra collision information, that is produced by other task \\
\notion{Note 2}: This is achieved by directly subscribing to required tables, but in specific cases these table are designed to be \programmatic{joined} with existing ones \\
\notion{Note 3}: Table \codeline{aod::Mults} is produced by \codeline{o2-analysis-multiplicity-table} \\
\notion{Note 4}: We can use iterators from joins: \codeline{soa::Join<Collisions, Mults>::iterator}
\end{block}

\begin{block}{At the generator level?}
No such additional table for MC collisions, so let's just count particles in V0 acceptance
\begin{minted}[fontsize=\small,bgcolor=gray!20,breaklines]{cpp}
if ((particle.eta() > 2.7f and particle.eta() < 5.1f) or (particle.eta() > -3.7f and particle.eta() < -1.7f))
\end{minted}
\end{block}
\framebreak
\begin{block}{Combining workflows}
\begin{minted}[fontsize=\small]{shell}
% o2-analysis-multiplicity-table --configuration=json://example_5.json |\
    o2-analysistutorial-example-5 --configuration=json://example_5.json
\end{minted}

\textcolor{red}{For each workflow to be able to build correct topology, they all need to know \emph{all} process switches, the easiest way is to give the same (complete) configuration to each of them}
\vspace{0.2ex}
\end{block}
\begin{block}{Caveats}
\notion{Note 1}: We can no longer use filter on MC particles -- we need to count those in different $\eta$ ranges \\
\notion{Note 2}: We are looking at all MC collisions, even those that do not have reconstructed counterpart \\

Can we do better? See next example.
\end{block}

{\footnotesize Relevant tutorial:  \href{https://github.com/AliceO2Group/O2Physics/blob/master/Tutorials/src/multiplicityEventTrackSelection.cxx}{https://github.com/AliceO2Group/O2Physics/blob/master/Tutorials/src/multiplicityEventTrackSelection.cxx} \\
    Complete example: \href{https://github.com/aalkin/O2Physics/blob/hands-on-examples/Tutorials/src/example_5.cxx}{https://github.com/aalkin/O2Physics/blob/hands-on-examples/Tutorials/src/example\_5.cxx}
}
\end{frame}

\begin{frame}[fragile,allowframebreaks]{Example 6}
\begin{block}{Derived tables}
\textbf{Add a joinable table with multiplicity information for MC collisions (V0M only) \codeline{aod::MultsGen}}

\notion{Note 1}: Look at \href{https://github.com/AliceO2Group/O2Physics/blob/master/Common/DataModel/Multiplicity.h}{Common/DataModel/Multiplicity.h} \\
\notion{Note 2}: Creating a table requires an \emph{additional} task: our main task will subscribe to the result \\
\notion{Note 3}: To fill a table one uses \codeline{Produces<TableType> cursor;} in the definition of the producer \\
\notion{Note 4}: The \programmatic{cursor} is used to fill the table by calling \codeline{cursor(col1_value, col2_value, ...);} \\
\end{block}
\vspace{1em}
\begin{block}{Adding an extra task}
\begin{itemize}
    \item Declare the new table in the same source file\footnote{It is acceptable for tables that are not intended to be shared with other workflows, otherwise create a common header}
    \item Declare the task in the same source file: \codeline{struct PreTask {};}
    \item \textcolor{red}{Add it to \codeline{defineDataProcessing(ConfigContext const& cfgc)}}
\end{itemize}
\notion{Note 1}: This task can once again use a filter to select particles in V0M acceptance
\end{block}
\framebreak
\begin{block}{Using derived tables}
Derived tables are used \emph{in an exactly same way} as any other tables!

\begin{minted}[fontsize=\small]{cpp}
void processMC(
    soa::Join<aod::McCollisions, aod::MultsGen>::iterator const& mccollision,
    soa::Filtered<aod::McParticles> const& particles) {}
\end{minted}

\notion{Note 1}: We can once again use a filter to apply $\eta$ cut
\end{block}

{\footnotesize Relevant tutorial:  \href{https://github.com/AliceO2Group/O2Physics/blob/master/Tutorials/src/associatedExample.cxx}{https://github.com/AliceO2Group/O2Physics/blob/master/Tutorials/src/associatedExample.cxx} \\
    Complete example: \href{https://github.com/aalkin/O2Physics/blob/hands-on-examples/Tutorials/src/example_6.cxx}{https://github.com/aalkin/O2Physics/blob/hands-on-examples/Tutorials/src/example\_6.cxx}
}
\end{frame}

\begin{frame}[fragile,allowframebreaks]{Example 7}
    \begin{block}{Index tables}
    \textbf{Only fill MC collisions that have reconstructed counterpart}

    \notion{Note 1}: We have predefined table \codeline{aod::McCollisionsLabels}, joinable with \codeline{aod::Collisions} to go from reco to gen via \programmatic{index} \\
    \notion{Note 2}: To reverse (or combine) 1-to-1 index connections we have a special mechanism of \programmatic{index tables} \\
    \notion{Note 3}: To achieve our goal we need to build an index table that is joinable with MC collisions and contains corresponding index to reco collisions \\
    \notion{Note 4}: \codeline{DECLARE_INDEX_TABLE_USER(Name, Key, Desc, joinable_idx, idx2, ...);} \\
    \notion{Note 5}: We will need to declare index columns we want to have in this table \\
    \hspace{3em} \codeline{DECLARE_INDEX_COLUMN(Target, getter);} \\
    \notion{Note 6}: \textcolor{red}{Simplified macro requires as target table name without final \enquote{s}: for example \codeline{Collision}} \\
    \notion{Note 7}: We need collisions to be mached to MC collisions \emph{through} MC collisions table, therefore we need \codelines{using CollisionsWLabels = soa::Join<aod::Collisions, aod::McCollisionLabels>;}{\scriptsize} as a target for corresponding index column
    \end{block}
\framebreak
    \begin{block}{Solution}
\begin{minted}[fontsize=\small,breaklines]{cpp}
using CollisionsWLabels = soa::Join<aod::Collisions, aod::McCollisionLabels>;
namespace o2::aod::idx
{
    DECLARE_SOA_INDEX_COLUMN(CollisionsWLabel, collision);
    DECLARE_SOA_INDEX_COLUMN(McCollision, mccollision);
} // namespace o2::aod::idx
namespace o2::aod
{
    DECLARE_SOA_INDEX_TABLE_USER(MatchedMCRec, McCollisions, "MMCR", idx::McCollisionId, idx::CollisionsWLabelId)
} // namespace o2::aod
\end{minted}
    \end{block}
\framebreak
\begin{block}{Using index tables}
\notion{Note 1}: Non-exclusive index tables are \emph{joinable} with the tables, that are used for the first index column \\
\notion{Note 2}: In our case \codeline{soa::Join<aod::McCollisions, aod::MultsGen, aod::MatchedMCRec>} \\
\notion{Note 3}: \textcolor{red}{Index table should be always last in joins} \\
\notion{Note 4}: In non-exclusive index tables index may be unset: one should always check with \codeline{.has_<getter>()} method before trying to access it
\end{block}

\begin{block}{Hint}
\begin{minted}[fontsize=\scriptsize,breaklines]{cpp}
void processMC(
   soa::Join<aod::McCollisions, aod::MultsGen, aod::MatchedMCRec>::iterator const& matched,
   soa::Filtered<aod::McParticles> const& particles)
{
    if (!matched.has_collision()) {
        return;
    }
  // ...
}
\end{minted}
\end{block}

{\footnotesize Relevant tutorial:  \href{https://github.com/AliceO2Group/O2Physics/blob/master/Tutorials/src/associatedExample.cxx}{https://github.com/AliceO2Group/O2Physics/blob/master/Tutorials/src/associatedExample.cxx} \\
    Complete example: \href{https://github.com/aalkin/O2Physics/blob/hands-on-examples/Tutorials/src/example_7.cxx}{https://github.com/aalkin/O2Physics/blob/hands-on-examples/Tutorials/src/example\_7.cxx}
}
\end{frame}

\begin{frame}[fragile,allowframebreaks]{Example 5 with partitions}
    \begin{block}{Partitions}
\textbf{Is it possible to define several subsamples of a collection with filters?}\\

\notion{Note 1}: This is what \codeline{Partition<TableType>} is for \\
\notion{Note 2}: It is defined similarly to a \programmatic{Filter}, however it is independent of the process function arguments \\
\notion{Note 3}: \textcolor{red}{However the table it is defined for needs to be subscribed} \\
\notion{Note 4}: We can even include the \codeline{isPhysicalPrimary()} requirement \\
    \end{block}

\begin{block}{Solution}
\begin{minted}[fontsize=\scriptsize,breaklines]{cpp}
Partition<aod::McParticles> central = nabs(aod::mcparticle::eta) < etaCut && (aod::mcparticle::flags & (uint8_t)o2::aod::mcparticle::enums::PhysicalPrimary) == (uint8_t)o2::aod::mcparticle::enums::PhysicalPrimary;
Partition<aod::McParticles> v0m = ((aod::mcparticle::eta > 2.7f && aod::mcparticle::eta < 5.1f) || (aod::mcparticle::eta > -3.7f && aod::mcparticle::eta < -1.7f)) && (aod::mcparticle::flags & (uint8_t)o2::aod::mcparticle::enums::PhysicalPrimary) == (uint8_t)o2::aod::mcparticle::enums::PhysicalPrimary;
\end{minted}
\end{block}

{\footnotesize Relevant tutorial:  \href{https://github.com/AliceO2Group/O2Physics/blob/master/Tutorials/src/multiplicityEventTrackSelection.cxx}{https://github.com/AliceO2Group/O2Physics/blob/master/Tutorials/src/multiplicityEventTrackSelection.cxx} \\
    Complete example: \href{https://github.com/aalkin/O2Physics/blob/hands-on-examples/Tutorials/src/example_5_partitions.cxx}{https://github.com/aalkin/O2Physics/blob/hands-on-examples/Tutorials/src/example\_5\_partitions.cxx}
}
\end{frame}

\begin{frame}{The end}
    \vspace{0.15\textheight}
    \centering
    \Huge
    \textbf{Thank you!}
\end{frame}

\end{document}
