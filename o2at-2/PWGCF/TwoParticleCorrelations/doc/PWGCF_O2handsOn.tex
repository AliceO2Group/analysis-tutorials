\documentclass[14pt,aspectratio=169,t]{beamer}

\mode<presentation> {
%\usetheme{Hannover}

%\usecolortheme{albatross}
%\usecolortheme{beaver}
%\usecolortheme{beetle}
%\usecolortheme{crane}
%\usecolortheme{dolphin}
%\usecolortheme{dove}
%\usecolortheme{fly}
%\usecolortheme{lily}
%\usecolortheme{orchid}
\usecolortheme{rose}
%\usecolortheme{seagull}
%\usecolortheme{seahorse}
%\usecolortheme{whale}
%\usecolortheme{wolverine}

% To remove the footer line in all slides uncomment this line
%\setbeamertemplate{footline} 

% To replace the footer line in all
% slides with a simple slide count uncomment this line
\setbeamertemplate{footline}[page number] 
\setbeamertemplate{section in toc}[ball numbered] 

% To remove the navigation symbols
%from the bottom of all slides uncomment this line
\setbeamertemplate{navigation symbols}{} 
\usefonttheme{professionalfonts}


\setbeamertemplate{itemize items}[square] % if you want a square
\setbeamertemplate{itemize subitem}[circle] % if you wnat a circle
\setbeamertemplate{itemize subsubitem}[triangle] % if you want a triangle
}


\usepackage[utf8]{inputenc}
\usepackage{amsmath}
\usepackage{amsfonts}
\usepackage{amssymb}
\usepackage{amsthm}
\usepackage{slashed}

\usepackage{graphicx} % Allows including images
\usepackage{booktabs} % Allows \toprule, \midrule and \bottomrule in tables
\usepackage{caption}
\usepackage{subcaption}

\usepackage{pgfplots}
\usepgfplotslibrary{dateplot}
\pgfplotsset{compat=1.15}
\usepackage{xspace}

\usepackage{tikz}
%\usepackage{soul}
\usepackage[normalem]{ulem}

\usepackage[overlay,absolute]{textpos}
\setlength{\TPHorizModule}{1cm}
\setlength{\TPVertModule}{1cm}
\textblockorigin{10mm}{10mm}


\makeatletter
\def\beamer@andinst{\quad}
\def\beamer@and{\\[1em]}
\makeatother


\definecolor{carmine}{rgb}{0.59, 0.0, 0.09}


\title{\texorpdfstring{O2 Analysis tutorial\\Second edition\\
        PWGCF O2 hands-on\\-- O2 and O2Physics frameworks --}{}}
\date{April 27, 2023}

\graphicspath{
  {./IMAGES/}
  {/localthree/victor/CERNBOX/TWO_PARTICLE_CORRELATIONS/ANALYSISRESULTS/}
  {/localthree/victor/CERNBOX/TWO_PARTICLE_CORRELATIONS/ANALYSISRESULTS/AA_YEARLYANALYSIS/2020ANALYSIS/}
  {/localthree/victor/CERNBOX/TWO_PARTICLE_CORRELATIONS/ANALYSISRESULTS/AA_YEARLYANALYSIS/2021ANALYSIS/}
  {/localthree/victor/DOCUMENTS/PHD/THESIS/Figures/}
  {/localthree/victor/DOCUMENTS/PHD/REFERENCES/}
  {/localtwo/victor/PROJECTS/PAPERS/}
  {/localtwo/victor/PROJECTS/BALANCEFUNCTIONS/STARCODE/IMAGES/}
  {/localtwo/victor/PROJECTS/DUKEQCD/TESTSOSG/}
  {/localtwo/victor/PROJECTS/DUKEQCD/TESTSOSG/NOTEBOOKS/}
  {/localtwo/victor/PROJECTS/PAPERS/ALICEREVIEWPAPER/NOTEBOOKS/}
  {/localthree/victor/DOCUMENTS/POSTDOC/WSU/PWGCF/O2MC_CHALLENGE/IMAGES/}
  {/localthree/victor/DOCUMENTS/POSTDOC/WSU/FOLLOWUP/IMAGES/}
}

%\AtBeginSection[]
%{
%\begin{frame}
%\frametitle{Contents}
%\tableofcontents[currentsection]
%\end{frame}
%}

% to insert the section title slide
\AtBeginSection[]{
  \begin{frame}
  \vfill
  \centering
  \begin{beamercolorbox}[sep=8pt,center,shadow=true,rounded=true]{title}
    \usebeamerfont{title}\insertsectionhead\par%
  \end{beamercolorbox}
  \vfill
  \end{frame}
}


%remove the icon
\setbeamertemplate{bibliography item}{}

%remove line breaks
\setbeamertemplate{bibliography entry title}{}
\setbeamertemplate{bibliography entry location}{}
\setbeamertemplate{bibliography entry note}{}

\begin{document}
\setbeamercolor{postit}{fg=black,bg=yellow}

\begin{frame}
    \begin{textblock}{14}(0.0,1.0)
        \titlepage
    \end{textblock}
    
    % the logos
    \begin{textblock}{14}(0.0,6.0)
        \centering
        Víctor González
    \end{textblock}
    \begin{textblock}{12}(0.0,-0.7)
        \includegraphics[width=6cm]{LOGO-WSU-wsu_primary_horz_color}
    \end{textblock}
    \begin{textblock}{10}(11.3,-0.8)
        \includegraphics[width=1.5cm]{O2_logo_S_130822}
    \end{textblock}
    \begin{textblock}{10}(13.3,-0.8)
        \includegraphics[width=1.5cm]{LOGO-ALICE-2012-Jul-04-4_Color_Logo_CB}
    \end{textblock}
\end{frame}

\begin{frame}
  \frametitle{O2 and O2Physics frameworks}
  \begin{itemize}
    \item Main goals
    \begin{itemize}
      \item Not learning everything about O2 and O2Physics
      \item Get a taste of O2 and O2Physics
      \item Build and run your first O2Physics analysis task
      \item Learn why you do what you will do
      \item Learn where you will find documentation
      \item Learn where you will find help
      \item Learn where you will be able to continue learning
      \item Have fun
    \end{itemize}
  \end{itemize}
\end{frame}

\begin{frame}[fragile]
  \frametitle{O2 and O2Physics at a glance. The analysis task}
  \vspace{-0.1in}
  {\tiny \color{blue}
  \begin{verbatim}
#include "Framework/runDataProcessing.h"
#include "Framework/AnalysisTask.h"

using namespace o2;
using namespace o2::framework;

struct firsto2physics {
  void init(InitContext&) {
  }

  void process(aod::Collisions const&, aod::Tracks const&) {
    LOGF(info, "Hello word");
  }
};

WorkflowSpec defineDataProcessing(ConfigContext const& cfgc) {
  return WorkflowSpec{adaptAnalysisTask<firsto2physics>(cfgc)};
}
  \end{verbatim}
  }
  \vspace{-0.25in}
  {\tiny \color{black}
  \begin{verbatim}
[O2Physics/latest] ~/SANDBOX $> o2-analysis-cf-firsto2physics --aod-file AO2D_MCPbPb_T233_R244983.root > execution.log    
  \end{verbatim}}
  \vspace{-0.25in}
  {\tiny \color{red}
  \begin{verbatim}
  \end{verbatim}}
  
  \begin{textblock}{6}(7.0,0.0)
    {\tiny\color{blue}
    \begin{verbatim}

o2physics_add_dpl_workflow(firsto2physics
                    SOURCES firsto2physics.cxx
                    PUBLIC_LINK_LIBRARIES O2::Framework 
                                          O2Physics::AnalysisCore
                                          O2Physics::PWGCFCore
                    COMPONENT_NAME Analysis)
    }
    \end{verbatim}}
  \end{textblock}

\end{frame}

\begin{frame}
  \frametitle{Where everything begins. AOD tables}
  \begin{textblock}{4}(0.5,-0.2)
    \begin{figure}
    \centering
      \includegraphics[scale=0.40,keepaspectratio=true,clip=true,trim=10pt 0pt 30pt 8pt]
      {HANDSON/DataFrames}
    \end{figure}
  \end{textblock}
  \begin{textblock}{4}(5.5,-0.2)
    \begin{figure}
    \centering
      \includegraphics[scale=0.40,keepaspectratio=true,clip=true,trim=8pt 0pt 14pt 195pt]
      {HANDSON/Tables}
    \end{figure}
  \end{textblock}
  \begin{textblock}{4}(10.5,-1.2)
    \begin{figure}
    \centering
      \includegraphics[scale=0.40,keepaspectratio=true,clip=true,trim=12pt 4pt 30pt 310pt]
      {HANDSON/CollisionsTable}
    \end{figure}
  \end{textblock}
  \begin{textblock}{4}(10.5,3.5)
    \begin{figure}
    \centering
      \includegraphics[scale=0.40,keepaspectratio=true,clip=true,trim=15pt 2pt 25pt 24pt]
      {HANDSON/TracksTable}
    \end{figure}
  \end{textblock}
\end{frame}

\begin{frame}
  \frametitle{Where everything is documented}
  \framesubtitle{\href{https://aliceo2group.github.io/analysis-framework/docs/}{https://aliceo2group.github.io/analysis-framework/docs/}}
  \begin{textblock}{15}(0.0,0.3)
    \begin{figure}
    \centering
      \includegraphics[scale=0.20,keepaspectratio=true,clip=true,trim=3pt 4pt 10pt 4pt]
      {HANDSON/O2DocumentationMain}
    \end{figure}
  \end{textblock}
\end{frame}

\begin{frame}
  \frametitle{Where everything is documented. Collisions table}
  \begin{textblock}{15}(0.0,-0.3)
    \begin{figure}
    \centering
      \includegraphics[scale=0.23,keepaspectratio=true,clip=true,trim=3pt 4pt 10pt 4pt]
      {HANDSON/O2DocumentationCollisionsTable}
    \end{figure}
  \end{textblock}
\end{frame}

\begin{frame}
  \frametitle{Where everything is documented. Tracks table}
  \begin{textblock}{15}(0.0,-0.3)
    \begin{figure}
    \centering
      \includegraphics[scale=0.23,keepaspectratio=true,clip=true,trim=3pt 30pt 10pt 4pt]
      {HANDSON/O2DocumentationTracksTable}
    \end{figure}
  \end{textblock}
\end{frame}

\begin{frame}
  \frametitle{Where actually everything is done}
  \begin{textblock}{2.5}(0.0,-0.0)
    \begin{center}
      {\small \color{blue}O2 tree}
    \end{center}
    \vspace{-0.3in}
    \begin{figure}
    \centering
      \includegraphics[scale=0.23,keepaspectratio=true,clip=true,trim=13pt 0pt 1pt 0pt]
      {HANDSON/O2Tree}
    \end{figure}
  \end{textblock}
  \begin{textblock}{2.8}(2.5,-0.0)
    \begin{center}
      {\small \color{blue}O2Physics tree}
    \end{center}
    \vspace{-0.3in}
    \begin{figure}
    \centering
      \includegraphics[scale=0.23,keepaspectratio=true,clip=true,trim=10pt 0pt 1pt 8pt]
      {HANDSON/O2PhysicsTree}
    \end{figure}
  \end{textblock}
  \begin{textblock}{2.8}(6.0,-0.0)
    \begin{center}
      {\small \color{blue}Data model}
    \end{center}
    \vspace{-0.5in}
    \begin{figure}
    \centering
      \includegraphics[scale=0.23,keepaspectratio=true,clip=true,trim=0pt 0pt 1pt 8pt]
      {HANDSON/O2DataModel}
    \end{figure}
  \end{textblock}
  \begin{textblock}{2.8}(10.5,-0.0)
    \begin{center}
      {\small \color{blue}PWGCF tasks}
    \end{center}
    \vspace{-0.5in}
    \begin{figure}
    \centering
      \includegraphics[scale=0.23,keepaspectratio=true,clip=true,trim=0pt 0pt 0pt 8pt]
      {HANDSON/PWGCFTasks}
    \end{figure}
  \end{textblock}
\end{frame}

\begin{frame}[fragile]
  \frametitle{Where actually everything is done. Collisions table}
  {\tiny\color{red}
  \begin{verbatim}
namespace collision
{
DECLARE_SOA_INDEX_COLUMN(BC, bc);                              //! Most probably BC to where this collision has occured
DECLARE_SOA_COLUMN(PosX, posX, float);                         //! X Vertex position in cm
DECLARE_SOA_COLUMN(PosY, posY, float);                         //! Y Vertex position in cm
DECLARE_SOA_COLUMN(PosZ, posZ, float);                         //! Z Vertex position in cm
DECLARE_SOA_COLUMN(CovXX, covXX, float);                       //! Vertex covariance matrix
DECLARE_SOA_COLUMN(CovXY, covXY, float);                       //! Vertex covariance matrix
DECLARE_SOA_COLUMN(CovXZ, covXZ, float);                       //! Vertex covariance matrix
DECLARE_SOA_COLUMN(CovYY, covYY, float);                       //! Vertex covariance matrix
DECLARE_SOA_COLUMN(CovYZ, covYZ, float);                       //! Vertex covariance matrix
DECLARE_SOA_COLUMN(CovZZ, covZZ, float);                       //! Vertex covariance matrix
DECLARE_SOA_COLUMN(Flags, flags, uint16_t);                    //! Run 2: see CollisionFlagsRun2 | Run 3: see Vertex::Flags
DECLARE_SOA_COLUMN(Chi2, chi2, float);                         //! Chi2 of vertex fit
DECLARE_SOA_COLUMN(NumContrib, numContrib, uint16_t);          //! Number of tracks used for the vertex
DECLARE_SOA_COLUMN(CollisionTime, collisionTime, float);       //! Collision time in ns relative to BC stored in bc()
DECLARE_SOA_COLUMN(CollisionTimeRes, collisionTimeRes, float); //! Resolution of collision time
} // namespace collision

DECLARE_SOA_TABLE(Collisions, "AOD", "COLLISION", //! Time and vertex information of collision
                  o2::soa::Index<>, collision::BCId,
                  collision::PosX, collision::PosY, collision::PosZ,
                  collision::CovXX, collision::CovXY, collision::CovXZ, collision::CovYY, collision::CovYZ, collision::CovZZ,
                  collision::Flags, collision::Chi2, collision::NumContrib,
                  collision::CollisionTime, collision::CollisionTimeRes);

using Collision = Collisions::iterator;
    
  \end{verbatim}}
\end{frame}

\begin{frame}[fragile]
  \frametitle{Where actually everything is done. Tracks table (I)}
  \vspace{-0.05in}
  {\tiny\color{red}
  \begin{verbatim}
namespace track
{
// TRACKPAR TABLE definition
DECLARE_SOA_INDEX_COLUMN(Collision, collision);    //! Collision to which this track belongs
DECLARE_SOA_COLUMN(TrackType, trackType, uint8_t); //! Type of track. See enum TrackTypeEnum
DECLARE_SOA_COLUMN(X, x, float);                   //!
DECLARE_SOA_COLUMN(Alpha, alpha, float);           //!
DECLARE_SOA_COLUMN(Y, y, float);                   //!
DECLARE_SOA_COLUMN(Z, z, float);                   //!
DECLARE_SOA_COLUMN(Snp, snp, float);               //!
DECLARE_SOA_COLUMN(Tgl, tgl, float);               //!
DECLARE_SOA_COLUMN(Signed1Pt, signed1Pt, float);   //! (sign of charge)/Pt in c/GeV. Use pt() and sign() instead
DECLARE_SOA_EXPRESSION_COLUMN(Phi, phi, float,     //! Phi of the track, in radians within [0, 2pi)
........
DECLARE_SOA_EXPRESSION_COLUMN(Eta, eta, float, //! Pseudorapidity
........
DECLARE_SOA_EXPRESSION_COLUMN(Pt, pt, float, //! Transverse momentum of the track in GeV/c
........
DECLARE_SOA_DYNAMIC_COLUMN(Sign, sign, //! Charge: positive: 1, negative: -1
........
DECLARE_SOA_DYNAMIC_COLUMN(Px, px, //! Momentum in x-direction in GeV/c
........
DECLARE_SOA_DYNAMIC_COLUMN(Py, py, //! Momentum in y-direction in GeV/c
........
DECLARE_SOA_DYNAMIC_COLUMN(Pz, pz, //! Momentum in z-direction in GeV/c
........
DECLARE_SOA_EXPRESSION_COLUMN(P, p, float, //! Momentum in Gev/c
........
} // namespace track

  \end{verbatim}}
\end{frame}


\begin{frame}[fragile]
  \frametitle{Where actually everything is done. Tracks table (II)}
  {\tiny\color{red}
  \begin{verbatim}
DECLARE_SOA_TABLE_FULL(StoredTracks, "Tracks", "AOD", "TRACK", //! On disk version of the Track table
                       o2::soa::Index<>, track::CollisionId, track::TrackType,
                       track::X, track::Alpha,
                       track::Y, track::Z, track::Snp, track::Tgl,
                       track::Signed1Pt,
                       track::Px<track::Signed1Pt, track::Snp, track::Alpha>,
                       track::Py<track::Signed1Pt, track::Snp, track::Alpha>,
                       track::Pz<track::Signed1Pt, track::Tgl>,
                       track::Sign<track::Signed1Pt>);

DECLARE_SOA_EXTENDED_TABLE(Tracks, StoredTracks, "TRACK", //! Basic track properties
                           aod::track::Pt,
                           aod::track::P,
                           aod::track::Eta,
                           aod::track::Phi);

using Track = Tracks::iterator;
  \end{verbatim}}
\end{frame}

\begin{frame}[fragile]
  \frametitle{Tables}
  -- {\color{blue} Collected data are stored in memory as tables}\par
  -- {\color{blue} They are written once and read many times}\par
  -- {\color{blue} Most efficient memory layout}\par
  -- {\color{blue} Never copied}\par
  \vspace{0.2in}
  -- {\color{blue} Different table operations: \verb!blue!, \textit{grouping}, ...}\par
  -- {\color{blue} Different ways to subscribe to a table}
  \begin{itemize}
    \item The whole table: {\color{blue}\verb|Collisions|}, {\color{blue}\verb|Tracks|}
    \item Row level ({\color{blue}\verb|iterator|}): {\color{blue}\verb|Collision|}, {\color{blue}\verb|Track|}
  \end{itemize}  
\end{frame}


\begin{frame}[fragile]
  \frametitle{Your first O2Physics analysis task}
  \vspace{-0.2in}
  {\scriptsize \color{blue}
  \begin{verbatim}
#include "Framework/runDataProcessing.h"
#include "Framework/AnalysisTask.h"

using namespace o2;
using namespace o2::framework;

struct firsto2physics {
  void init(InitContext&) {
  }

  void process(aod::Collisions const&, aod::Tracks const&) {
    LOGF(info, "Hello word");
  }
};

WorkflowSpec defineDataProcessing(ConfigContext const& cfgc) {
  return WorkflowSpec{adaptAnalysisTask<firsto2physics>(cfgc)};
}
  \end{verbatim}
  }
  \vspace{-0.3in}
  What about configure the task and how to produce/store the results
\end{frame}

\section{Let's start doing things}

\begin{frame}[fragile]
  \frametitle{Configuring your task}
  \vspace{-0.0in}
  {\scriptsize \color{blue}
  \begin{verbatim}
struct firsto2physics {
  Configurable<float> cfgZVtxCut = {"zvtxcut", 7.0, "Vertex z cut. Default 7 cm"};
  Configurable<float> cfgPtCutMin = {"minpt", 0.2, "Minimum accepted track pT. Default 0.2 GeV"};
  Configurable<float> cfgPtCutMax = {"maxpt", 5.0, "Maximum accepted track pT. Default 5.0 GeV"};
  Configurable<float> cfgEtaCut = {"etacut", 0.8, "Eta cut. Default 0.8"};

  void init(InitContext&)
  {
  }
  ...................
  \end{verbatim}}
  \vspace{-0.3in}
  \begin{itemize}
    \small
    \item Full set of different configurables (documentation and tutorials)
    \item Accessible from
    \begin{itemize}
      \item Command line (\verb|--zvtxcut|)
      \item \verb|.json| configuration file
      \item hyperloop
    \end{itemize}
  \end{itemize}
\end{frame}

\begin{frame}[fragile]
  \frametitle{Building and registering your task outputs}
  \vspace{-0.0in}
  {\tiny \color{blue}
  \begin{verbatim}
  ...................
  Configurable<float> cfgEtaCut = {"etacut", 0.8, "Eta cut. Default 0.8"};

  HistogramRegistry registry{"MyFirstO2TaskResults", {}, OutputObjHandlingPolicy::AnalysisObject, false, true};
  void init(InitContext&)
  {
    registry.add("before/reco/zvtx", "vtx_{#it{z}}", kTH1F, {{300, -15.0, 15.0, "vtx_{#it{z}} (cm)"}});
    registry.add("before/reco/multiplicity", "V0M multiplicity class", kTH1F, {{100, 0.0, 100.0, "V0M multiplicity (%)"}});
    registry.add("before/reco/pt", "Charged particles #it{p}_{T}", kTH1F, {{150, 0.0, 15.0, "#it{p}_{T} (GeV/#it{c})"}});
    registry.add("before/reco/eta", "Charged particles #eta", kTH1F, {{400, -2.0, 2.0, "#eta"}});
    registry.add("before/reco/phi", "Charged particles #varphi", kTH1F, {{360, 0.0, constants::math::TwoPI, "#varphi"}});

    registry.addClone("before/reco/", "after/reco/");
    registry.get<TH1>(HIST("after/reco/zvtx"))->GetXaxis()->SetRangeUser(-cfgZVtxCut, cfgZVtxCut);
    registry.get<TH1>(HIST("after/reco/pt"))->GetXaxis()->SetRangeUser(cfgPtCutMin, cfgPtCutMax);
    registry.get<TH1>(HIST("after/reco/eta"))->GetXaxis()->SetRangeUser(-cfgEtaCut, cfgEtaCut);
  }
  ...................
  \end{verbatim}}
  \vspace{-0.3in}
  \begin{itemize}
    \small
    \item \verb|HistogramRegistry| is the most convenient and versatile way
    \item There are others (documentation and tutorials)
  \end{itemize}
\end{frame}



\begin{frame}[fragile]
  \frametitle{Selecting collisions and tracks}
  \framesubtitle{Place holders for the time being}
  {\scriptsize\color{blue}
  \begin{verbatim}
template <typename CollisionInstance>
bool isCollisionAccepted(CollisionInstance const& col)
{
  bool accepted = true;

  return accepted;
}

template <typename TrackInstance>
bool isTrackAccepted(TrackInstance const& track)
{
  bool accepted = true;

  return accepted;
}

  \end{verbatim}}
  \vspace{-0.5in}
  \begin{itemize}
    \small
    \item Templated for versatility and not repeated code
    \item Only handle what is given and not something else
  \end{itemize}
\end{frame}

\begin{frame}[fragile]
  \frametitle{Start receiving data and filling output results}
  {\tiny\color{blue}
  \vspace{-0.25in}
  \begin{verbatim}
using CollisionInstance = soa::Join<aod::Collisions, aod::CentV0Ms>::iterator;

void process(CollisionInstance const& collision, aod::Tracks const& tracks)
{
  registry.fill(HIST("before/reco/zvtx"), collision.posZ());
  registry.fill(HIST("before/reco/multiplicity"), collision.centV0M());

  if (isCollisionAccepted(collision)) {

    registry.fill(HIST("after/reco/zvtx"), collision.posZ());
    registry.fill(HIST("after/reco/multiplicity"), collision.centV0M());

    for (auto const& track : tracks) {
      registry.fill(HIST("before/reco/pt"), track.pt());
      registry.fill(HIST("before/reco/eta"), track.eta());
      registry.fill(HIST("before/reco/phi"), track.phi());

      if (isTrackAccepted(track)) {
        registry.fill(HIST("after/reco/pt"), track.pt());
        registry.fill(HIST("after/reco/eta"), track.eta());
        registry.fill(HIST("after/reco/phi"), track.phi());
      }
    }
  }
}
  \end{verbatim}}
  \vspace{-0.4in}
  \begin{itemize}
    \small
    \item Table operations: \verb|Join| and \textit{grouping}
    \item Subscribe to one collision$+$multiplicity information and to associated tracks
  \end{itemize}
\end{frame}

\section{Time to compile it\\Time to run it\\So how was it?}

\begin{frame}[fragile]
  \frametitle{Why it failed}
  \small
  {\tiny\color{red}
  \vspace{-0.2in}
  \begin{verbatim}
[24744:internal-dpl-aod-reader]: [12:08:46][ERROR] Exception caught: 
 Couldn't get TTree "DF_2449832441401065001/O2centv0m" from "AO2D_MCPbPb_T233_R244983.root". 
Please check 
 https://aliceo2group.github.io/analysis-framework/docs/troubleshooting/treenotfound.html 
 for more information. 
  \end{verbatim}}
  \vspace{-0.2in}
  Subscribed to \texttt{centrality} table whose producer task was not invoked
  {\tiny\color{black}
  \vspace{-0.1in}
  \begin{verbatim}
    o2-analysis-cf-firsto2physics --aod-file AO2D_MCPbPb_T233_R244983.root | o2-analysis-centrality-table > execution.log
  \end{verbatim}}
  {\tiny\color{red}
  \vspace{-0.4in}
  \begin{verbatim}
[26737:internal-dpl-aod-reader]: [12:58:57][ERROR] Exception caught: 
 Couldn't get TTree "DF_2449832441401065001/O2mult" from "AO2D_MCPbPb_T233_R244983.root"
  \end{verbatim}}
  \vspace{-0.2in}
  \texttt{centrality} task needs \texttt{multiplicity} table whose producer was not invoked
  {\tiny\color{black}
  \vspace{-0.1in}
  \begin{verbatim}
    o2-analysis-cf-firsto2physics --aod-file AO2D_MCPbPb_T233_R244983.root |  
      o2-analysis-centrality-table | 
      o2-analysis-multiplicity-table > execution.log
  \end{verbatim}}
  {\tiny\color{red}
  \vspace{-0.4in}
  \begin{verbatim}
[29393:internal-dpl-aod-reader]: [13:15:15][ERROR] Exception caught: 
 Couldn't get TTree "DF_2449832441401065001/O2timestamps" from "AO2D_MCPbPb_T233_R244983.root"
  \end{verbatim}}
  \vspace{-0.2in}
  \texttt{multiplicity} task needs \texttt{timestamp} table whose producer was not invoked
  {\tiny\color{black}
  \vspace{-0.1in}
  \begin{verbatim}
    o2-analysis-cf-firsto2physics --aod-file AO2D_MCPbPb_T233_R244983.root |  
     o2-analysis-centrality-table | 
     o2-analysis-multiplicity-table | 
     o2-analysis-timestamp  > execution.log
  \end{verbatim}}
\end{frame}

\begin{frame}[fragile]
  \frametitle{The configuration file}
  \framesubtitle{Use as source \texttt{dpl-config.json} and rename it}
  \begin{textblock}{6}(0.0,0.2)
    {\tiny\color{blue}
    \vspace{-0.0in}
    \begin{verbatim}
{
    "internal-dpl-clock": "",
    "internal-dpl-aod-reader": {
        "time-limit": "0",
        "orbit-offset-enumeration": "0",
        "orbit-multiplier-enumeration": "0",
        "start-value-enumeration": "0",
        "end-value-enumeration": "-1",
        "step-value-enumeration": "1",
        "aod-file": "AO2D_MCPbPb_T233_R244983.root"
    },
    "internal-dpl-aod-index-builder": "",
    "internal-dpl-aod-spawner": "",
    "timestamp-task": {
        "verbose": "false",
        "rct-path": "RCT\/RunInformation",
        "start-orbit-path": "GRP\/StartOrbit",
        "ccdb-url": "http:\/\/alice-ccdb.cern.ch",
        "isRun2MC": "false"
    },
    \end{verbatim}}
  \end{textblock}
  \begin{textblock}{6}(7.0,0.2)
    {\tiny\color{blue}
    \vspace{-0.0in}
    \begin{verbatim}
    "multiplicity-table": {
        "processRun2": "true",
        "processRun3": "false"
    },
    "centrality-table": {
        "estV0M": "-1",
        "estRun2SPDtks": "-1",
        "estRun2SPDcls": "-1",
        "estRun2CL0": "-1",
        "estRun2CL1": "-1"
    },
    "firsto2physics": {
        "zvtxcut": "7",
        "minpt": "0.200000003",
        "maxpt": "5",
        "etacut": "0.800000012"
    },
    "internal-dpl-aod-global-analysis-file-sink": "",
    "internal-dpl-aod-writer": ""
}
    \end{verbatim}}
  \end{textblock}
  \begin{textblock}{6}(0.0,5.5)
    {\tiny\color{black}
    \vspace{-0.0in}
    \begin{verbatim}
      o2-analysis-cf-firsto2physics --configuration json://handsonconfig.json |  
        o2-analysis-centrality-table --configuration json://handsonconfig.json | 
        o2-analysis-multiplicity-table --configuration json://handsonconfig.json | 
        o2-analysis-timestamp  --configuration json://handsonconfig.json > execution.log
    \end{verbatim}}
  \end{textblock}
\end{frame}

\section{Have a look at your results}

\begin{frame}[fragile]
  \frametitle{Let's prepare for generator level analysis}
  \framesubtitle{Output, and collisions and particle selections (place holders for the time being)}
  {\tiny\color{blue}
  \begin{verbatim}
  void init(InitContext&)
  {
    ................
    registry.addClone("before/reco/", "before/gen/");
    registry.addClone("after/reco/", "after/gen/");
  }

  template <typename CollisionInstance>
  bool isGenCollisionAccepted(CollisionInstance const& col)
  {
    bool accepted = true;

    return accepted;
  }

  template <typename ParticleInstance>
  bool isParticleAccepted(ParticleInstance const& particle)
  {
    bool accepted = true;

    return accepted;
  }
  \end{verbatim}}
\end{frame}

\begin{frame}[fragile]
  \frametitle{Subscribing to generator level tables}
  \framesubtitle{Signature of \texttt{process} subscribed to \texttt{Collisions}$+$\texttt{CentV0Ms} and \texttt{Tracks}}
  \vspace{-0.0in}
  \small
  {\scriptsize\color{blue}
  \begin{verbatim}
void process(CollisionInstance const& collision, aod::Tracks const& tracks)
{
  ..........
}
  \end{verbatim}}
  \vspace{-0.2in}
  Keep signature, change name, and incorporate \texttt{PROCESS\_SWITCH} mechanism
  {\scriptsize\color{blue}
  \begin{verbatim}
void processReco(CollisionInstance const& collision, aod::Tracks const& tracks)
{
  ..........
}
PROCESS_SWITCH(firsto2physics, processReco, "Process reconstructed collisions and tracks", true);
  \end{verbatim}}
\end{frame}

\begin{frame}[fragile]
  \frametitle{Start receiving/processing generator level data}
  {\tiny\color{blue}
  \vspace{-0.0in}
  \begin{verbatim}
  void processGen(aod::McCollision const& gencollision, aod::McParticles const& particles)
  {
    registry.fill(HIST("before/gen/zvtx"), gencollision.posZ());

    if (isGenCollisionAccepted(gencollision)) {

      registry.fill(HIST("after/gen/zvtx"), gencollision.posZ());

      for (auto const& particle : particles) {
        registry.fill(HIST("before/gen/pt"), particle.pt());
        registry.fill(HIST("before/gen/eta"), particle.eta());
        registry.fill(HIST("before/gen/phi"), particle.phi());

        if (isParticleAccepted(particle)) {
          registry.fill(HIST("after/gen/pt"), particle.pt());
          registry.fill(HIST("after/gen/eta"), particle.eta());
          registry.fill(HIST("after/gen/phi"), particle.phi());
        }
      }
    }
  }
  PROCESS_SWITCH(firsto2physics, processGen, "Process reconstructed collisions and tracks", false);
  \end{verbatim}}
  \vspace{-0.3in}
  \begin{itemize}
    \small
    \item Default is just reconstructed level
  \end{itemize}
\end{frame}

\section{Time to compile it\\Time to run it\\Time to have a look at the results}

\begin{frame}[fragile]
  \frametitle{Revisiting the configuration file}
  \framesubtitle{First have a look at \texttt{dpl-config.json}}
  \begin{textblock}{6}(0.0,0.2)
    {\tiny\color{blue}
    \vspace{-0.0in}
    \begin{verbatim}
{
    "internal-dpl-clock": "",
    "internal-dpl-aod-reader": {
        "time-limit": "0",
        "orbit-offset-enumeration": "0",
        "orbit-multiplier-enumeration": "0",
        "start-value-enumeration": "0",
        "end-value-enumeration": "-1",
        "step-value-enumeration": "1",
        "aod-file": "AO2D_MCPbPb_T233_R244983.root"
    },
    "internal-dpl-aod-spawner": "",
    "internal-dpl-aod-index-builder": "",
    "timestamp-task": {
        "verbose": "false",
        "rct-path": "RCT\/RunInformation",
        "start-orbit-path": "GRP\/StartOrbit",
        "ccdb-url": "http:\/\/alice-ccdb.cern.ch",
        "isRun2MC": "true"
    },
    \end{verbatim}}
  \end{textblock}
  \begin{textblock}{6}(7.0,0.2)
    {\tiny\color{blue}
    \vspace{-0.0in}
    \begin{verbatim}
    "multiplicity-table": {
        "processRun2": "true",
        "processRun3": "false"
    },
    "centrality-table": {
        "estV0M": "-1",
        "estRun2SPDtks": "-1",
        "estRun2SPDcls": "-1",
        "estRun2CL0": "-1",
        "estRun2CL1": "-1"
    },
    "firsto2physics": {
        "zvtxcut": "7",
        "minpt": "0.200000003",
        "maxpt": "5",
        "etacut": "0.800000012",
        "processReco": "true",
        "processGen": "false"
    },
    "internal-dpl-aod-global-analysis-file-sink": "",
    "internal-dpl-aod-writer": ""
}
    \end{verbatim}}
  \end{textblock}
  \begin{textblock}{14}(0.0,6.2)
    \begin{itemize}
      \small
      \item Incorporate the two new {\color{blue}\texttt{process...}} lines into your configuration file
      \item Change {\color{blue} \verb|"processGen": "true"|}, run and have a look at the results
    \end{itemize}

  \end{textblock}
\end{frame}

\begin{frame}[fragile]
  \frametitle{Detector level handling. Tracks particle labels}
  \vspace{0.1in}
  -- {\color{blue} The treatment of MC detector level must be the same as the treatment of reconstructed data}
  \begin{itemize}
    \small
    \item But bad tracks (with negative labels) shall be discarded
  \end{itemize}
  \vspace{0.1in}
  -- {\color{blue} \verb|McTrackLabel| table needs to be joined to the  \verb|Tracks| table}\par
  \vspace{0.1in}
  -- {\color{blue} Different signature}
  \begin{itemize}
    \small
    \item Repeat \verb|processReco| with different signature
    \item Exploit the \verb|template| mechanism
  \end{itemize}
\end{frame}

\begin{frame}[fragile]
  \frametitle{Generic unified reconstructed data processing}
  {\tiny\color{blue}
  \vspace{-0.1in}
  \begin{verbatim}
  template <typename CollisionInstance, typename TracksListInstance>
  void processReconstructed(CollisionInstance const& collision, TracksListInstance const& tracks)
  {
    registry.fill(HIST("before/reco/zvtx"), collision.posZ());
    registry.fill(HIST("before/reco/multiplicity"), collision.centV0M());

    if (isCollisionAccepted(collision)) {

      registry.fill(HIST("after/reco/zvtx"), collision.posZ());
      registry.fill(HIST("after/reco/multiplicity"), collision.centV0M());

      for (auto const& track : tracks) {
        registry.fill(HIST("before/reco/pt"), track.pt());
        registry.fill(HIST("before/reco/eta"), track.eta());
        registry.fill(HIST("before/reco/phi"), track.phi());

        if (isTrackAccepted(track)) {
          registry.fill(HIST("after/reco/pt"), track.pt());
          registry.fill(HIST("after/reco/eta"), track.eta());
          registry.fill(HIST("after/reco/phi"), track.phi());
        }
      }
    }
  }
  \end{verbatim}}
  \vspace{-0.3in}
  \begin{itemize}
    \small
    \item No subscriptions here (only in \verb|process...|)
    \item Generic collision table/s and generic tracks tables
  \end{itemize}
\end{frame}

\begin{frame}[fragile]
  \frametitle{Expand process switching options. Check for labels}
  {\tiny\color{blue}
  \vspace{-0.1in}
  \begin{verbatim}
  using CollisionInstance = soa::Join<aod::Collisions, aod::CentV0Ms>::iterator;

  void processRecoData(CollisionInstance const& collision, aod::Tracks const& tracks)
  {
    processReconstructed(collision, tracks);
  }
  PROCESS_SWITCH(firsto2physics, processRecoData, "Process data reconstructed collisions and tracks", true);

  void processDetectorLevel(CollisionInstance const& collision, soa::Join<aod::Tracks, aod::McTrackLabels> const& tracks)
  {
    processReconstructed(collision, tracks);
  }
  PROCESS_SWITCH(firsto2physics, processDetectorLevel, "Process MC detector level collisions and tracks", false);
  \end{verbatim}}
  {\tiny\color{blue}
  \vspace{-0.4in}
  \begin{verbatim}
  template <typename TrackInstance>
  bool isTrackAccepted(TrackInstance const& track)
  {
    if (track.mcParticleId() < 0) {
      return false;
    }
    bool accepted = true;

    return accepted;
  }
  \end{verbatim}}
  \vspace{-0.4in}
  \begin{itemize}
    \small
    \item It does not compile!
  \end{itemize}
\end{frame}

\begin{frame}[fragile]
  \frametitle{Help the compiler!}
  -- {\color{blue} Identify the code sections which apply to certain tables}\par
  -- {\color{blue} In this case even the associated particle can be recovered}\par
  {\tiny\color{blue}
  \vspace{-0.0in}
  \begin{verbatim}
  template <typename TrackInstance>
  bool isTrackAccepted(TrackInstance const& track)
  {
    if constexpr (framework::has_type_v<aod::mctracklabel::McParticleId, typename TrackInstance::all_columns>) {
      if (track.mcParticleId() < 0) {
        return false;
      }
      auto particle = track.mcParticle();
    }
    bool accepted = true;

    return accepted;
  }
  \end{verbatim}}
  \vspace{-0.2in}
  -- {\color{blue} But the \verb|McParticles| table needs to be subscribed}\par
  {\tiny\color{blue}
  \vspace{-0.0in}
  \begin{verbatim}
  void processDetectorLevel(CollisionInstance const& collision, 
                            soa::Join<aod::Tracks, aod::McTrackLabels> const& tracks, 
                            aod::McParticles const&)
  {
    processReconstructed(collision, tracks);
  }
  PROCESS_SWITCH(firsto2physics, processDetectorLevel, "Process MC detector level collisions and tracks", false);
  \end{verbatim}}
\end{frame}

\section{Time to compile again\\Time to run}

\begin{frame}[fragile]
  \frametitle{Revisiting the configuration file}
  \framesubtitle{First have a look at \texttt{dpl-config.json}}
  \begin{textblock}{6}(0.0,0.2)
    {\tiny\color{blue}
    \vspace{-0.0in}
    \begin{verbatim}
{
    "internal-dpl-clock": "",
    "internal-dpl-aod-reader": {
        "time-limit": "0",
        "orbit-offset-enumeration": "0",
        "orbit-multiplier-enumeration": "0",
        "start-value-enumeration": "0",
        "end-value-enumeration": "-1",
        "step-value-enumeration": "1",
        "aod-file": "AO2D_MCPbPb_T233_R244983.root"
    },
    "internal-dpl-aod-spawner": "",
    "internal-dpl-aod-index-builder": "",
    "timestamp-task": {
        "verbose": "false",
        "rct-path": "RCT\/RunInformation",
        "start-orbit-path": "GRP\/StartOrbit",
        "ccdb-url": "http:\/\/alice-ccdb.cern.ch",
        "isRun2MC": "true"
    },
    \end{verbatim}}
  \end{textblock}
  \begin{textblock}{6}(7.0,0.2)
    {\tiny\color{blue}
    \vspace{-0.0in}
    \begin{verbatim}
    "multiplicity-table": {
        "processRun2": "true",
        "processRun3": "false"
    },
    "centrality-table": {
        "estV0M": "-1",
        "estRun2SPDtks": "-1",
        "estRun2SPDcls": "-1",
        "estRun2CL0": "-1",
        "estRun2CL1": "-1"
    },
    "firsto2physics": {
        "zvtxcut": "7",
        "minpt": "0.200000003",
        "maxpt": "5",
        "etacut": "0.800000012",
        "processRecoData": "true",
        "processDetectorLevel": "false",
        "processGen": "false"
    },
    "internal-dpl-aod-global-analysis-file-sink": "",
    "internal-dpl-aod-writer": ""
}    \end{verbatim}}
  \end{textblock}
  \begin{textblock}{14}(0.0,6.2)
    \begin{itemize}
      \small
      \item Incorporate the two new {\color{blue}\texttt{process...}} lines into your configuration file
      \item Set {\scriptsize\color{blue} \verb|"processRecoData": "false"|, \verb|"processDetectorLevel": "true"|, \verb|"processGen": "true"|}
    \end{itemize}
  \end{textblock}
\end{frame}

\section{Homework\\Incorporate event selection\\Incorporate track selection}

\begin{frame}
  \frametitle{How to keep learning}
  \small
  -- {\color{blue} O2 documentation}\par
  \begin{itemize}
    \vspace{-0.1in}
    \item \href{https://aliceo2group.github.io/analysis-framework/}{https://aliceo2group.github.io/analysis-framework/}
  \end{itemize}
  -- {\color{blue} O2Physics tutorials} 
  \begin{itemize}
    \vspace{-0.1in}
    \item \href{https://aliceo2group.github.io/analysis-framework/docs/tutorials/}{https://aliceo2group.github.io/analysis-framework/docs/tutorials/}
    \vspace{-0.1in}
    \item \texttt{alice/O2Physics/Tutorials}
  \end{itemize}
  -- {\color{blue} Your colleagues O2Physics tasks} 
  \begin{itemize}
    \vspace{-0.1in}
    \item \texttt{alice/O2Physics/PWG...}
  \end{itemize}
  -- {\color{blue} O2 Analysis mattermost channel}\par
  \begin{itemize}
    \vspace{-0.1in}
    \item \href{https://mattermost.web.cern.ch/alice/channels/o2-analysis}{https://mattermost.web.cern.ch/alice/channels/o2-analysis}
  \end{itemize}
  -- {\color{blue} Hyperloop Operation mattermost channel}\par
  \begin{itemize}
    \vspace{-0.1in}
    \item \href{https://mattermost.web.cern.ch/alice/channels/o2-hyperloop-operation}{https://mattermost.web.cern.ch/alice/channels/o2-hyperloop-operation}
  \end{itemize}

   
\end{frame}

\end{document}

