\documentclass[14pt,aspectratio=169,t]{beamer}

\mode<presentation> {
%\usetheme{Hannover}

%\usecolortheme{albatross}
%\usecolortheme{beaver}
%\usecolortheme{beetle}
%\usecolortheme{crane}
%\usecolortheme{dolphin}
%\usecolortheme{dove}
%\usecolortheme{fly}
%\usecolortheme{lily}
%\usecolortheme{orchid}
\usecolortheme{rose}
%\usecolortheme{seagull}
%\usecolortheme{seahorse}
%\usecolortheme{whale}
%\usecolortheme{wolverine}

% To remove the footer line in all slides uncomment this line
%\setbeamertemplate{footline} 

% To replace the footer line in all
% slides with a simple slide count uncomment this line
\setbeamertemplate{footline}[page number] 
\setbeamertemplate{section in toc}[ball numbered] 

% To remove the navigation symbols
%from the bottom of all slides uncomment this line
\setbeamertemplate{navigation symbols}{} 
\usefonttheme{professionalfonts}


\setbeamertemplate{itemize items}[square] % if you want a square
\setbeamertemplate{itemize subitem}[circle] % if you wnat a circle
\setbeamertemplate{itemize subsubitem}[triangle] % if you want a triangle
}


\usepackage[utf8]{inputenc}
\usepackage{amsmath}
\usepackage{amsfonts}
\usepackage{amssymb}
\usepackage{amsthm}
\usepackage{slashed}

\usepackage{graphicx} % Allows including images
\usepackage{booktabs} % Allows \toprule, \midrule and \bottomrule in tables
\usepackage{caption}
\usepackage{subcaption}

\usepackage{pgfplots}
\usepgfplotslibrary{dateplot}
\pgfplotsset{compat=1.15}
\usepackage{xspace}

\usepackage{tikz}
%\usepackage{soul}
\usepackage[normalem]{ulem}

\usepackage[overlay,absolute]{textpos}
\setlength{\TPHorizModule}{1cm}
\setlength{\TPVertModule}{1cm}
\textblockorigin{10mm}{10mm}


\makeatletter
\def\beamer@andinst{\quad}
\def\beamer@and{\\[1em]}
\makeatother


\definecolor{carmine}{rgb}{0.59, 0.0, 0.09}


\title{\texorpdfstring{O2 Analysis tutorial\\Second edition\\
        PWGCF O2 hands-on\\-- Correlations analysis framework --}{}}
\date{April 27, 2023}

\graphicspath{
  {./IMAGES/}
  {/localthree/victor/CERNBOX/TWO_PARTICLE_CORRELATIONS/ANALYSISRESULTS/}
  {/localthree/victor/CERNBOX/TWO_PARTICLE_CORRELATIONS/ANALYSISRESULTS/AA_YEARLYANALYSIS/2020ANALYSIS/}
  {/localthree/victor/CERNBOX/TWO_PARTICLE_CORRELATIONS/ANALYSISRESULTS/AA_YEARLYANALYSIS/2021ANALYSIS/}
  {/localthree/victor/DOCUMENTS/PHD/THESIS/Figures/}
  {/localthree/victor/DOCUMENTS/PHD/REFERENCES/}
  {/localtwo/victor/PROJECTS/PAPERS/}
  {/localtwo/victor/PROJECTS/BALANCEFUNCTIONS/STARCODE/IMAGES/}
  {/localtwo/victor/PROJECTS/DUKEQCD/TESTSOSG/}
  {/localtwo/victor/PROJECTS/DUKEQCD/TESTSOSG/NOTEBOOKS/}
  {/localtwo/victor/PROJECTS/PAPERS/ALICEREVIEWPAPER/NOTEBOOKS/}
  {/localthree/victor/DOCUMENTS/POSTDOC/WSU/PWGCF/O2MC_CHALLENGE/IMAGES/}
  {/localthree/victor/DOCUMENTS/POSTDOC/WSU/FOLLOWUP/IMAGES/}
}

%\AtBeginSection[]
%{
%\begin{frame}
%\frametitle{Contents}
%\tableofcontents[currentsection]
%\end{frame}
%}

% to insert the section title slide
\AtBeginSection[]{
  \begin{frame}
  \vfill
  \centering
  \begin{beamercolorbox}[sep=8pt,center,shadow=true,rounded=true]{title}
    \usebeamerfont{title}\insertsectionhead\par%
  \end{beamercolorbox}
  \vfill
  \end{frame}
}

%remove the icon
\setbeamertemplate{bibliography item}{}

%remove line breaks
\setbeamertemplate{bibliography entry title}{}
\setbeamertemplate{bibliography entry location}{}
\setbeamertemplate{bibliography entry note}{}

\begin{document}
\setbeamercolor{postit}{fg=black,bg=yellow}

\begin{frame}
    \begin{textblock}{14}(0.0,1.0)
        \titlepage
    \end{textblock}
    
    % the logos
    \begin{textblock}{14}(0.0,4.8)
        \centering
        {\color{blue}\small (Jan Fiete's seminal \texttt{AliAnalysisTaskPhiCorrelations})}\\
        \vspace{0.3in}Víctor González
    \end{textblock}
    \begin{textblock}{12}(0.0,-0.7)
        \includegraphics[width=6cm]{LOGO-WSU-wsu_primary_horz_color}
    \end{textblock}
    \begin{textblock}{10}(11.3,-0.8)
        \includegraphics[width=1.5cm]{O2_logo_S_130822}
    \end{textblock}
    \begin{textblock}{10}(13.3,-0.8)
        \includegraphics[width=1.5cm]{LOGO-ALICE-2012-Jul-04-4_Color_Logo_CB}
    \end{textblock}
\end{frame}

\begin{frame}[fragile]
  \frametitle{Correlations containers}
  -- {\color{blue}Reconstructed step based multidimensional histograms}
  \begin{itemize}
    \item Each configured reconstruction step its own histograms set 
    \item Single particle magnitudes
    \item Two-particle magnitudes
  \end{itemize}
  \vspace{0.1in}
  -- {\color{blue}$p_{\rm T}$ differential analysis}\par
  \vspace{0.1in}
  -- {\color{blue}Trigger -- associated analysis}\par
  \vspace{0.1in}
  -- {\color{blue}Reconstruction step efficiency extraction}\par
  \vspace{0.1in}
  -- {\color{blue}Check} {\small \verb|alice/O2Physics/PWGCF/Core/CorrelationContainer.h|}\par
  \vspace{0.1in}
  {\small\color{blue}
  \begin{verbatim}
      #include "PWGCF/Core/CorrelationContainer.h"
  \end{verbatim}}
\end{frame}

\begin{frame}[fragile]
  \frametitle{Mixed events support}
  -- {\color{blue} There are many options.} Check 
  \begin{itemize}
    \small
    \item \texttt{O2Physics/Tutorials/src/eventMixing.cxx} and 
    \item \texttt{O2Physics/Tutorials/src/eventMixingValidation.cxx}
  \end{itemize}
  \vspace{0.1in}
  -- {\color{blue}All of them allow to classify the objects to mix}
  \vspace{-0.1in}
  {\scriptsize\color{blue}
  \begin{verbatim}
  // define our collision classes
  std::vector<double> 
    vtxBinsEdges{VARIABLE_WIDTH, -7.0f, -5.0f, -3.0f, -1.0f, 1.0f, 3.0f, 5.0f, 7.0f};
  std::vector<double> 
    multBinsEdges{VARIABLE_WIDTH, 0.0f, 5.0f, 10.0f, 20.0f, 30.0f, 40.0f, 50.0, 100.1f};
  // define the mixing binding to our collision classes
  ColumnBinningPolicy<aod::collision::PosZ, aod::cent::CentRun2V0M> 
    bindingOnVtxAndMult{{vtxBinsEdges, multBinsEdges}, true}; 
        // true is for 'ignore overflows' (true by default)
  // define the pairing we expect
  SameKindPair<soa::Filtered<soa::Join<aod::Collisions, aod::EvSels, aod::CentRun2V0Ms>>, 
               soa::Filtered<soa::Join<aod::Tracks, aod::TrackSelection>>, 
               ColumnBinningPolicy<aod::collision::PosZ, aod::cent::CentRun2V0M>> 
    pair{bindingOnVtxAndMult, 5, -1}; 
        // indicates that 5 events should be mixed and under/overflow (-1) to be ignored
  \end{verbatim}}
\end{frame}

\begin{frame}[fragile]
  \frametitle{Processing mixed events}  
  \vspace{-0.1in}
  {\scriptsize\color{blue}
  \begin{verbatim}
// filter out non interesting
Filter collisionZVtxFilter = nabs(aod::collision::posZ) < cfgZVtxCut;
Filter trackFilter = (nabs(aod::track::eta) < cfgEtaCut) && 
                      (aod::track::pt > cfgPtCutMin) && 
                      (aod::track::pt < cfgPtCutMax) &&
                      (requireGlobalTrackInFilter() || 
                      (aod::track::isGlobalTrackSDD == (uint8_t) true));


void processMixed(soa::Filtered<soa::Join<aod::Collisions, aod::EvSels, aod::CentRun2V0Ms>> const& 
                        collisions,
                  soa::Filtered<soa::Join<aod::Tracks, aod::TrackSelection>> const& 
                        tracks)
{
  for (auto& [collision1, tracks1, collision2, tracks2] : pair) {
    ..................................  
    ..................................
  }
}
PROCESS_SWITCH(firstcorrelations, processMixed, "Process mixed events", true);


  \end{verbatim}}
\end{frame}

\begin{frame}[fragile]
  \frametitle{Pair cuts}
  \framesubtitle{Functions which cut on particle pairs (decays, conversions, pair proximity)}
  -- {\color{blue} Currently support $\gamma$, ${\rm K}^{0}$, $\Lambda$, $\phi$ and $\rho$}\par
  \vspace{0.1in}
  -- {\color{blue} Own set of control histograms in a passed \texttt{HistogramRegistry}}\par
  \vspace{0.1in}
  -- {\color{blue} Fully configurable}\par
  \vspace{0.1in}
  -- {\color{blue}Check} {\small \texttt{alice/O2Physics/PWGCF/Core/PairCuts.h}}\par
  \vspace{0.1in}
  {\small\color{blue}
  \begin{verbatim}
        #include "PWGCF/Core/PairCuts.h"
  \end{verbatim}}
\end{frame}

\begin{frame}[fragile]
  \frametitle{Your first O2Physics correlations analysis task}
  \vspace{-0.1in}
  {\scriptsize \color{blue}
  \begin{verbatim}
#include "Framework/runDataProcessing.h"
#include "Framework/AnalysisTask.h"

using namespace o2;
using namespace o2::framework;

struct firstcorrelations {
  void init(InitContext&)
  {
  }

  void process(aod::Collisions const&, aod::Tracks const&)
  {
    LOGF(info, "Hello word");
  }
};

WorkflowSpec defineDataProcessing(ConfigContext const& cfgc)
{
  return WorkflowSpec{adaptAnalysisTask<firstcorrelations>(cfgc)};
}
  \end{verbatim}
  }
\end{frame}

\section{Let's start doing things}

\begin{frame}[fragile]
  \frametitle{Configuring your task}
  \vspace{-0.15in}
  {\tiny\color{blue}
  \begin{verbatim}
static constexpr float cfgPairCutDefaults[1][5] = {{-1, -1, -1, -1, -1}};

struct firstcorrelations {
  Configurable<float> cfgZVtxCut = {"zvtxcut", 7.0, "Vertex z cut. Default 7 cm"};
  Configurable<float> cfgPtCutMin = {"minpt", 0.2, "Minimum accepted track pT. Default 0.2 GeV"};
  Configurable<float> cfgPtCutMax = {"maxpt", 5.0, "Maximum accepted track pT. Default 5.0 GeV"};
  Configurable<float> cfgEtaCut = {"etacut", 0.8, "Eta cut. Default 0.8"};

  Configurable<LabeledArray<float>> cfgPairCut{"cfgPairCut", 
                                               {cfgPairCutDefaults[0], 
                                                5, 
                                                {"Photon", "K0", "Lambda", "Phi", "Rho"}}, 
                                                "Pair cuts on various particles"};

  ConfigurableAxis axisVertex{"axisVertex", {7, -7, 7}, "vertex axis for histograms"};
  ConfigurableAxis axisDeltaPhi{"axisDeltaPhi", {72, -constants::math::PIHalf, constants::math::PIHalf * 3}, 
                                "delta phi axis for histograms"};
  ConfigurableAxis axisDeltaEta{"axisDeltaEta", {40, -2, 2}, "delta eta axis for histograms"};
  ConfigurableAxis axisPtTrigger{"axisPtTrigger", {VARIABLE_WIDTH, 0.5, 1.0, 1.5, 2.0, 3.0, 4.0, 6.0, 10.0}, 
                                 "pt trigger axis for histograms"};
  ConfigurableAxis axisPtAssoc{"axisPtAssoc", {VARIABLE_WIDTH, 0.5, 1.0, 1.5, 2.0, 3.0, 4.0, 6.0}, 
                               "pt associated axis for histograms"};
  ConfigurableAxis axisMultiplicity{"axisMultiplicity", {VARIABLE_WIDTH, 0, 5, 10, 20, 30, 40, 50, 100.1}, 
                                    "multiplicity / centrality axis for histograms"};
  ConfigurableAxis axisVertexEfficiency{"axisVertexEfficiency", {10, -10, 10}, "vertex axis for efficiency histograms"};
  ConfigurableAxis axisEtaEfficiency{"axisEtaEfficiency", {20, -1.0, 1.0}, "eta axis for efficiency histograms"};
  ConfigurableAxis axisPtEfficiency{"axisPtEfficiency", 
                                    {VARIABLE_WIDTH, 0.5, 0.6, 0.7, 0.8, 0.9, 1.0, 1.25, 1.5, 1.75, 2.0, 2.25, 
                                    2.5, 2.75, 3.0, 3.25, 3.5, 3.75, 4.0, 4.5, 5.0, 6.0, 7.0, 8.0}, 
                                    "pt axis for efficiency histograms"};
  void init(InitContext&)
  \end{verbatim}}
\end{frame}

\begin{frame}[fragile]
  \frametitle{Track your headers and usings and move forward}
  \vspace{-0.15in}
  {\scriptsize\color{blue}
  \begin{verbatim}
#include "Framework/runDataProcessing.h"
#include "Framework/AnalysisTask.h"
#include "Framework/AnalysisDataModel.h"
#include "Framework/ASoAHelpers.h"
#include "Framework/HistogramRegistry.h"
#include "Framework/RunningWorkflowInfo.h"
#include "CommonConstants/MathConstants.h"
#include "Common/DataModel/EventSelection.h"
#include "Common/DataModel/TrackSelectionTables.h"
#include "Common/DataModel/Centrality.h"
#include "PWGCF/Core/CorrelationContainer.h"
#include "PWGCF/Core/PairCuts.h"

using namespace o2;
using namespace o2::framework;
using namespace o2::framework::expressions;
  \end{verbatim}}
  \vspace{-0.2in}
  \begin{itemize}
    \small
    \item Compile, run, execute and have a look at the `dpl-config.json`
  \end{itemize}
  {\tiny \color{black}
  \vspace{-0.1in}
  \begin{verbatim}
        o2-analysis-cf-firstcorrelations --aod-file AO2D_MCPbPb_T233_R244983.root > execution.log
  \end{verbatim}}
\end{frame}

\begin{frame}[fragile]
  \frametitle{The configuration file}
  \framesubtitle{Use as source \texttt{dpl-config.json} and rename it}
  \begin{textblock}{6}(0.0,0.2)
    {\tiny\color{blue}
    \vspace{-0.0in}
    \begin{verbatim}
{
    ..............................
    "firstcorrelations": {
        "zvtxcut": "7",
        "minpt": "0.200000003",
        "maxpt": "5",
        "etacut": "0.800000012",
        "cfgPairCut": {
            "labels_rows": "",
            "labels_cols": [
                "Photon",
                "K0",
                "Lambda",
                "Phi",
                "Rho"
            ],
            "values": [
                [
                    "-1",
                    "-1",
                    "-1",
                    "-1",
                    "-1"
                ]
            ]
        },
    ..............................
}    
    \end{verbatim}}
  \end{textblock}
  \begin{textblock}{6}(7.0,-0.2)
    {\tiny\color{blue}
    \vspace{-0.0in}
    \begin{verbatim}
    \end{verbatim}}
  \end{textblock}
\end{frame}

\begin{frame}[fragile]
  \frametitle{Building and registering your task outputs and objects}
  \begin{textblock}{16}(-0.4,-0.3)
    {\tiny \color{blue}
    \begin{verbatim}
OutputObj<CorrelationContainer> same{"sameEvent"};
OutputObj<CorrelationContainer> mixed{"mixedEvent"};
HistogramRegistry registry{"registry"};
PairCuts mPairCuts;
bool doPairCuts = false;

void init(InitContext&)
{
  registry.add("yields", "centrality vs pT vs eta", {HistType::kTH3F, {{100, 0, 100, "centrality"}, 
                                                                       {40, 0, 20, "p_{T}"}, {100, -2, 2, "#eta"}}});
  registry.add("etaphi", "centrality vs eta vs phi", {HistType::kTH3F, {{100, 0, 100, "centrality"}, 
                                                                        {100, -2, 2, "#eta"}, {200, 0, 2 * M_PI, "#varphi"}}});
  const int maxMixBin = axisMultiplicity->size() * axisVertex->size();
  registry.add("eventcount", "bin", {HistType::kTH1F, {{maxMixBin + 2, -2.5, -0.5 + maxMixBin, "bin"}}});
  mPairCuts.SetHistogramRegistry(&registry);
  if (cfgPairCut->get("Photon") > 0 || cfgPairCut->get("K0") > 0 || cfgPairCut->get("Lambda") > 0 || 
      cfgPairCut->get("Phi") > 0 || cfgPairCut->get("Rho") > 0) {
    mPairCuts.SetPairCut(PairCuts::Photon, cfgPairCut->get("Photon"));
    mPairCuts.SetPairCut(PairCuts::K0, cfgPairCut->get("K0"));
    mPairCuts.SetPairCut(PairCuts::Lambda, cfgPairCut->get("Lambda"));
    mPairCuts.SetPairCut(PairCuts::Phi, cfgPairCut->get("Phi"));
    mPairCuts.SetPairCut(PairCuts::Rho, cfgPairCut->get("Rho"));
    doPairCuts = true;
  }
  std::vector<AxisSpec> axisList = {{axisDeltaEta, "#Delta#eta"}, {axisPtAssoc, "p_{T} (GeV/c)"}, {axisPtTrigger, "p_{T} (GeV/c)"},
                                    {axisMultiplicity, "multiplicity / centrality"}, {axisDeltaPhi, "#Delta#varphi (rad)"},
                                    {axisVertex, "z-vtx (cm)"}, {axisEtaEfficiency, "#eta"}, {axisPtEfficiency, "p_{T} (GeV/c)"},
                                      {axisVertexEfficiency, "z-vtx (cm)"}};
  same.setObject(new CorrelationContainer("sameEvent", "sameEvent", axisList));
  mixed.setObject(new CorrelationContainer("mixedEvent", "mixedEvent", axisList));
}
    \end{verbatim}}
  \end{textblock}
\end{frame}

\begin{frame}[fragile]
  \frametitle{Put in place collisions classification}
  
  \vspace{-0.0in}
  {\tiny\color{blue}
  \begin{verbatim}
std::vector<double> vtxBinsEdges{VARIABLE_WIDTH, -7.0f, -5.0f, -3.0f, -1.0f, 1.0f, 3.0f, 5.0f, 7.0f};
std::vector<double> multBinsEdges{VARIABLE_WIDTH, 0.0f, 5.0f, 10.0f, 20.0f, 30.0f, 40.0f, 50.0, 100.1f};
ColumnBinningPolicy<aod::collision::PosZ, aod::cent::CentRun2V0M> 
  bindingOnVtxAndMult{{vtxBinsEdges, multBinsEdges}, true}; // true is for 'ignore overflows' (true by default)
SameKindPair<soa::Filtered<soa::Join<aod::Collisions, aod::EvSels, aod::CentRun2V0Ms>>, 
              soa::Filtered<soa::Join<aod::Tracks, aod::TrackSelection>>, 
              ColumnBinningPolicy<aod::collision::PosZ, aod::cent::CentRun2V0M>> 
  pair{bindingOnVtxAndMult, 5, -1}; // indicates that 5 events should be mixed and under/overflow (-1) to be ignored

  \end{verbatim}}
\end{frame}



\begin{frame}[fragile]
  \frametitle{Tracks and collisions selection and subscription}
  \vspace{-0.1in}
  {\tiny\color{blue}
  \begin{verbatim}
Filter collisionZVtxFilter = nabs(aod::collision::posZ) < cfgZVtxCut;
Filter trackFilter = (nabs(aod::track::eta) < cfgEtaCut) && (aod::track::pt > cfgPtCutMin) && (aod::track::pt < cfgPtCutMax) &&
                      (requireGlobalTrackInFilter() || (aod::track::isGlobalTrackSDD == (uint8_t) true));


void processSame(soa::Filtered<soa::Join<aod::Collisions, aod::EvSels, aod::CentRun2V0Ms>>::iterator const& collision, 
                 soa::Filtered<soa::Join<aod::Tracks, aod::TrackSelection>> const& tracks)
{
}
PROCESS_SWITCH(firstcorrelations, processSame, "Process same event", true);

void processMixed(soa::Filtered<soa::Join<aod::Collisions, aod::EvSels, aod::CentRun2V0Ms>>& collisions, 
                  soa::Filtered<soa::Join<aod::Tracks, aod::TrackSelection>> const& tracks)
{
}
PROCESS_SWITCH(firstcorrelations, processMixed, "Process mixed events", true);
  \end{verbatim}}
  \vspace{-0.2in}
  \begin{itemize}
    \small
    \item Compile, run, execute and have a look at the `dpl-config.json`
  \end{itemize}
  \vspace{-0.4in}
  \begin{textblock}{16}(1.5,5.2)
    {\tiny \color{black}
    \vspace{-0.1in}
    \begin{verbatim}
o2-analysis-cf-firstcorrelations --configuration json://handsonconfig.json |  
  o2-analysis-centrality-table --configuration json://hdsonconfig.json | 
  o2-analysis-multiplicity-table --configuration json://handsonconfig.json | 
  o2-analysis-timestamp  --configuration json://handsonconfig.json | 
  o2-analysis-trackselection  --configuration json://handsonconfig.json | 
  o2-analysis-trackextension  --configuration json://handsonconfig.json | 
  o2-analysis-event-selection  --configuration json://handsonconfig.json > execution.log
    \end{verbatim}}
  \end{textblock}
\end{frame}

\begin{frame}[fragile]
  \frametitle{The configuration file}
  \framesubtitle{Get input from \texttt{dpl-config.json}}
  \begin{textblock}{6}(0.5,0.4)
    {\scriptsize\color{blue}
    \vspace{-0.0in}
    \begin{verbatim}
{
    .................
    "firstcorrelations": {
       .................
       "processSame": "true",
       "processMixed": "true"
    },
    .................
}    
    \end{verbatim}}
  \end{textblock}
  \begin{textblock}{6}(7.0,0.2)
    {\tiny\color{blue}
    \vspace{-0.0in}
    \begin{verbatim}
    \end{verbatim}}
  \end{textblock}
\end{frame}

\begin{frame}[fragile]
  \frametitle{Do the processing (I)}
  \begin{textblock}{6}(0.0,0.0)
    {\tiny\color{blue}
    \vspace{-0.0in}
    \begin{verbatim}
  template <typename TCollision, typename TTracks>
  void fillQA(TCollision collision, float centrality, TTracks tracks)
  {
    for (auto& track1 : tracks) {
      registry.fill(HIST("yields"), centrality, track1.pt(), track1.eta());
      registry.fill(HIST("etaphi"), centrality, track1.eta(), track1.phi());
    }
  }

  template <typename TTarget, typename TCollision>
  bool fillCollision(TTarget target, TCollision collision, float centrality)
  {
    target->fillEvent(centrality, CorrelationContainer::kCFStepAll);

    if (!collision.alias()[kINT7] || !collision.sel7()) {
      return false;
    }

    target->fillEvent(centrality, CorrelationContainer::kCFStepReconstructed);

    return true;
  }
    \end{verbatim}}
  \end{textblock}
  \begin{textblock}{6}(7.0,0.2)
    {\tiny\color{blue}
    \vspace{-0.0in}
    \begin{verbatim}
    \end{verbatim}}
  \end{textblock}
\end{frame}

\begin{frame}[fragile]
  \frametitle{Do the processing (II)}
  \begin{textblock}{6}(0.0,0.0)
    {\tiny\color{blue}
    \vspace{-0.0in}
    \begin{verbatim}
  template <typename TTarget, typename TTracks>
  void fillCorrelations(TTarget target, TTracks tracks1, TTracks tracks2, float centrality, float posZ)
  {
    for (auto& track1 : tracks1) {
      target->getTriggerHist()->Fill(CorrelationContainer::kCFStepReconstructed, track1.pt(), centrality, posZ, 1.0);

      for (auto& track2 : tracks2) {
        if (track1 == track2) {
          continue;
        }
        if (doPairCuts && mPairCuts.conversionCuts(track1, track2)) {
          continue;
        }
        float deltaPhi = track1.phi() - track2.phi();
        if (deltaPhi > 1.5f * PI) {
          deltaPhi -= TwoPI;
        }
        if (deltaPhi < -PIHalf) {
          deltaPhi += TwoPI;
        }

        target->getPairHist()->Fill(CorrelationContainer::kCFStepReconstructed,
                                    track1.eta() - track2.eta(), track2.pt(), track1.pt(), centrality, deltaPhi, posZ,
                                    1.0);
      }
    }
  }
    \end{verbatim}}
  \end{textblock}
  \begin{textblock}{6}(7.0,0.2)
    {\tiny\color{blue}
    \vspace{-0.0in}
    \begin{verbatim}
    \end{verbatim}}
  \end{textblock}
\end{frame}


\begin{frame}[fragile]
  \frametitle{Process same events}
  \begin{textblock}{6}(0.0,0.4)
    {\tiny\color{blue}
    \vspace{-0.0in}
    \begin{verbatim}
  void processSame(soa::Filtered<soa::Join<aod::Collisions, aod::EvSels, aod::CentRun2V0Ms>>::iterator const& collision, 
                   soa::Filtered<soa::Join<aod::Tracks, aod::TrackSelection>> const& tracks)
  {
    const auto centrality = collision.centV0M();

    if (fillCollision(same, collision, centrality) == false) {
      return;
    }
    registry.fill(HIST("eventcount"), -2);
    fillQA(collision, centrality, tracks);
    fillCorrelations(same, tracks, tracks, centrality, collision.posZ());
  }
    \end{verbatim}}
  \end{textblock}
  \begin{textblock}{6}(7.0,0.2)
    {\tiny\color{blue}
    \vspace{-0.0in}
    \begin{verbatim}
    \end{verbatim}}
  \end{textblock}
\end{frame}

\begin{frame}[fragile]
  \frametitle{Process mixed events}
  \begin{textblock}{6}(0.0,0.4)
    {\tiny\color{blue}
    \vspace{-0.2in}
    \begin{verbatim}
  void processMixed(soa::Filtered<soa::Join<aod::Collisions, aod::EvSels, aod::CentRun2V0Ms>>& collisions,
                    soa::Filtered<soa::Join<aod::Tracks, aod::TrackSelection>> const& tracks)
  {
    for (auto& [collision1, tracks1, collision2, tracks2] : pair) {

      if (fillCollision(mixed, collision1, collision1.centRun2V0M()) == false) {
        continue;
      }
      registry.fill(HIST("eventcount"), 1);

      fillCorrelations(mixed, tracks1, tracks2, collision1.centRun2V0M(), collision1.posZ());
    }
  }
    \end{verbatim}}
  \end{textblock}
  \begin{textblock}{6}(7.0,0.2)
    {\tiny\color{blue}
    \vspace{-0.0in}
    \begin{verbatim}
    \end{verbatim}}
  \end{textblock}
\end{frame}

\begin{frame}
  \frametitle{How to keep learning}
  \small
  -- {\color{blue} O2 documentation}\par
  \begin{itemize}
    \vspace{-0.1in}
    \item \href{https://aliceo2group.github.io/analysis-framework/}{https://aliceo2group.github.io/analysis-framework/}
  \end{itemize}
  -- {\color{blue} O2Physics tutorials} 
  \begin{itemize}
    \vspace{-0.1in}
    \item \href{https://aliceo2group.github.io/analysis-framework/docs/tutorials/}{https://aliceo2group.github.io/analysis-framework/docs/tutorials/}
    \vspace{-0.1in}
    \item \texttt{alice/O2Physics/Tutorials}
  \end{itemize}
  -- {\color{blue} Your colleagues O2Physics tasks} 
  \begin{itemize}
    \vspace{-0.1in}
    \item \texttt{alice/O2Physics/PWG...}
  \end{itemize}
  -- {\color{blue} O2 Analysis mattermost channel}\par
  \begin{itemize}
    \vspace{-0.1in}
    \item \href{https://mattermost.web.cern.ch/alice/channels/o2-analysis}{https://mattermost.web.cern.ch/alice/channels/o2-analysis}
  \end{itemize}
  -- {\color{blue} Hyperloop Operation mattermost channel}\par
  \begin{itemize}
    \vspace{-0.1in}
    \item \href{https://mattermost.web.cern.ch/alice/channels/o2-hyperloop-operation}{https://mattermost.web.cern.ch/alice/channels/o2-hyperloop-operation}
  \end{itemize}
\end{frame}
\end{document}

